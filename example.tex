% !TeX encoding = UTF-8
% !TeX program = xelatex
% !TeX spellcheck = en_US

\documentclass{cjc}
\usepackage{booktabs}
\usepackage{algorithm}
\usepackage{algorithmic}
\usepackage{siunitx}

\classsetup{
  % 配置里面不要出现空行
  title        = {题目},
  title*       = {Title},
  authors      = {
    author1 = {
      name         = {作者名},
      name*        = {NAME Name-Name},
      affiliations = {aff1},
      biography    = {性别,xxxx年生,学位(或目前学历),职称,是/否计算机学会(CCF)会员(提供会员号),主要研究领域为*****、****.},
      % 英文作者介绍内容包括:出生年, 学位(或目前学历), 职称, 主要研究领域(与中文作者介绍中的研究方向一致).
      biography*   = {Ph.D., asociate profesor. His/her research interests include ***, ***, and ***.},
      email        = {},
      phone-number = {……},  % 第1作者手机号码(投稿时必须提供,以便紧急联系,发表时会删除)
    },
    author2 = {
      name         = {作者名},
      name*        = {NAME Name},
      affiliations = {aff2, aff3},
      biography    = {性别,xxxx年生,学位(或目前学历),职称,是/否计算机学会(CCF)会员(提供会员号),主要研究领域为*****、****.},
      biography*   = {英文作者介绍内容包括:出生年, 学位(或目前学历), 职称, 主要研究领域(与中文作者介绍中的研究方向一致).},
      email        = {**************},
    },
    author3 = {
      name         = {作者},
      name*        = {NAME Name-Name},
      affiliations = {aff3},
      biography    = {性别,xxxx年生,学位(或目前学历),职称,是/否计算机学会(CCF)会员(提供会员号),主要研究领域为*****、****.},
      biography*   = {英文作者介绍内容包括:出生年, 学位(或目前学历), 职称, 主要研究领域(与中文作者介绍中的研究方向一致).},
      email        = {**************},
      % 通讯作者
      corresponding = true,
    },
  },
  % 论文定稿后,作者署名、单位无特殊情况不能变更。若变更,须提交签章申请,
  % 国家名为中国可以不写,省会城市不写省的名称,其他国家必须写国家名。
  affiliations = {
    aff1 = {
      name  = {单位全名\ 部门(系)全名, 市(或直辖市) 国家名\ 邮政编码},
      name* = {Department of ****, University, City ZipCode, Country},
    },
    aff2 = {
      name  = {单位全名\ 部门(系)全名, 市(或直辖市) 国家名\ 邮政编码},
      name* = {Department of ****, University, City ZipCode},
    },
    aff3 = {
      name  = {单位全名\ 部门(系)全名, 市(或直辖市) 国家名\ 邮政编码},
      name* = {Department of ****, University, City ZipCode, Country},
    },
  },
  abstract     = {
    中文摘要内容置于此处(英文摘要中要有这些内容),字体为小5号宋体。
    摘要贡献部分,要有数据支持,不要出现“...大大提高”、“...显著改善”等描述,
    正确的描述是“比…提高 X\%”、 “在…上改善 X\%”。
  },
  abstract*    = {Abstract (500英文单词,内容包含中文摘要的内容). },
  %中文关键字与英文关键字对应且一致,应有5-7个关键词,不要用英文缩写
  keywords     = {关键词, 关键词, 关键词, 关键词},
  keywords*    = {key word, key word, key word, key word},
  grants       = {
    本课题得到……基金中文完整名称(No.项目号)、
    ……基金中文完整名称(No.项目号)、
    ……基金中文完整名称(No.项目号)资助.
  },
  clc           = {TP393},
  doi           = {10.11897/SP.J.1016.2020.00001},  % 投稿时不提供DOI号
  received-date = {2019-08-10},  % 收稿日期
  revised-date  = {2019-10-19},  % 最终修改稿收到日期,投稿时不填写此项
  publish-date  = {2020-03-16},  % 出版日期
  page          = 512,
}

\newcommand\dif{\mathop{}\!\mathrm{d}}

% hyperref 总是在导言区的最后加载
\usepackage{hyperref}



\begin{document}

\maketitle{}


\section{引言}
  what is the research topic?\\
  a. backgroud information \\
  b. literature review\\
  what is the motivetion?\\
  theis statement\\
  A 宽泛-->研究话题(重要性)\\
  B 研究的空白\\










% 对投稿的基本要求:\\
% 稿件提交时的基本要求:\\
% (1)本模板中要求的各项内容正确齐全,无遗漏;\\
% (2)语句通顺,无中文、英文语法错误,易于阅读理解,符号使用正确,图、表清晰无误;
% (3)在学术、技术上,论文内容正确无误,各项内容确定。
%  \begin{equation}
%  \{
%  \'x = x_2 \label{eq1} 
%  \end{equation}
% \subsection{二级标题}
% \subsubsection{三级标题}
% 正文部分, 字体为5号宋体。
% 文件排版采用 TeX Live。
% 正文文字要求语句通顺,无语法错误,结构合理,条理清楚,不影响审稿人、读者阅读理解全文内容。以下几类问题请作者们特别注意:
% 1)文章题目应明确反映文章的思想和方法;文字流畅,表述清楚;
% 2)中文文字、英文表达无语法错误;
% 3)公式中无符号、表达式的疏漏,没有同一个符号表示两种意思的情况;
% 4)数学中使用的符号、函数名用斜体;
% 5)使用的量符合法定计量单位标准;
% 6)矢量为黑体,标量为白体;
% 7)变量或表示变化的量用斜体;
% 8)图表规范,量、线、序无误,位置正确(图表必须在正文中有所表述后出现,即…如图1所示)(注意纵、横坐标应有坐标名称和刻度值)。
% 9)列出的参考文献必须在文中按顺序引用,即参考文献顺序与引用顺序一致,各项信息齐全(格式见参考文献部分);
% 10)首次出现的缩写需写明全称,首次出现的符号需作出解释。
% 11)图的图例说明、坐标说明全部用中文或量符号。
% 12)图应为矢量图。
% 13)表中表头文字采用中文。
% 14)公式尺寸:
% 下标/上标:5.8磅
% 次下标/上标:4.5磅
% 符号:16磅
% 次符号:10.5磅
% 15)组合单位采用标准格式,如:“pJ/bit/m4”应为“\si{pJ/(bit.m^4)}”


% 1.符号命名一致性
% 2.符号第一次出现时要对其描述
% 3.等式的编号
% 4.不要灌水
\section{模型和数学知识}

  \subsection{智能汽车模型}
  \paragraph{}对于一排智能汽车(如fig.1)的首辆汽车,其动力学方程可以表示\:
  \begin{equation} \dot{x}_0(t)=v_0(t),\quad \dot{v}_0(t)=a_0(t) \end{equation}
  式中:$x_0(t)$为初始位置,$v_0(t)$和$a_0(t)$是速度和加速度,$a_0(t)$下面给出
  汽车$k$可以用如下三层非线性模型表示\:
  \begin{eqnarray}
    &\dot{x}_K(t) &= v_i(t) \nonumber \\
    &\dot{v}_K(t) &= a_i(t) \nonumber \\
    &\dot{a}_k(t) &=-\frac{1}{\tau_k}(a_i+\frac{\upsilon A_k C_{dk}}{2m_k} v_i(t)^2+\frac{t_{mk}}{m_k}) \nonumber \\
    &&\quad  -\frac{\upsilon A_k C_{dk} v_k(t) a_k}{m_k} + \frac{e_k(t)}{\tau_k m_k}+\epsilon _k(t)    
  \end{eqnarray}
  式中:$e_k (t)$是引擎/刹车的输入,$\tau _k$是引擎持续时间,$\upsilon$ 为空气比重,$m_k,A_k,C_{dk},t_{mk}$分别指的是第k辆车的质量,横截面阻尼系数,
  机械阻力,$\frac{\upsilon A_k C_{dk}}{2m_k}$指的是空气阻力,$\epsilon _k$代表由阵风或者路面情况而造成的额外干扰。\\
  令
  \begin{eqnarray}
    u_i &=& \frac{c_k(t)}{m_k} \nonumber \\
    f_k(v_k,a_k,t) &=& -\frac{1}{\tau _k}(a_i+\frac{\upsilon A_k C_{dk}}{2m_k} v_i(t)^2+\frac{e_{mk}}{m_k}) \nonumber \\
    &&\quad -\frac{\upsilon A_k C_{dk} v_k(t) a_k}{m_k}
    \end{eqnarray}
    将()带入()中,可得\:
    \begin{eqnarray}
      \dot{x}_k(t) &=& v_i(t) \nonumber \\
      \dot{v}_k(t) &=& a_i(t) \nonumber \\
      \dot{a}_K(t) &=& f_k(v_k,a_k,t) + \frac{u_k(t)}{\tau _K} + \epsilon _k(t)
    \end{eqnarray}
    在这里用$u_k(t)$来做输入,避免了汽车质量的影响。在现实当中机械阻力$t_{mk}$和空气阻尼$\frac{\upsilon A_k C_{dk}}{2m_k}$
    是无法获取,所以非线性函数$f_k(v_k,a_k,t)$是未知的,可以REFNN逼近。
    \paragraph{对{\color{red}制动器故障}中的误差和饱和特性建模}
    在实际情况下,部分的功率损失和偏差是汽车制动器故障主要的主要来源。其中功率损失主要由于磨损而偏差指的是老化或者
    外部压力是损耗。根据[],我们可采用如下模型\:
    \begin{equation}
      u_{ak} = \omega (t) u_k(t) + r_k(t) 
    \end{equation}
    其中$\omega (t)$指的是制动器的功效而$r_k(t)$值得是偏差。考虑到制动器故障来源以及制动器的饱和特性,可得\:
    \begin{eqnarray}
      &\dot{x}_k(t) =& v_i(t) \nonumber \\
      &\dot{v}_k(t) =& a_i(t) \nonumber \\
      &\dot{a}_k(t) =& f_k(v_k,a_k) \nonumber + \frac{\omega (t) sat(u_k(t)) + r_k(t)}{\tau _k} + \epsilon _k(t)
    \end{eqnarray}
    其中$sat(u_k(t))$代表非对称饱和特性,其表达式为\:
    \begin{equation}
      sat(u_k(t)) = 
      \begin{cases}
        u_{r,kmax}, & \mbox{if } u_k(t) > a_{r,k} \\
        g_{r,k}(u_k(t)) &\mbox{if }  0\leq u_k(t) \leq a_{r,k}\\
        g_{l,k}(u_k(t)) &\mbox{if } a_{l,k} \leq u_k(t) \leq 0\\
        u_{l,kmin} &\mbox{if } u_k(t) < a_{l,k}
      \end{cases}
    \end{equation}

    其中未知常数$u_{r,kmax} > 0 , u_{l,kmin} < 0$分别是制动器输出的上界和下界,$b_{r,k} > 0, b_{l,k}<0$是
    制动器的振幅,$g_{r,k}(u_k(t)),g_{l.k}(u_k(t))$是未知的非线性函数。参考[],可得以下等式\:
    \begin{equation}
      sat(u_k) = \chi _{uk}(t) u_k(t)
    \end{equation}
    其中$\chi _{uk}(t)$为饱和度指标函数。存在$\chi _k$满足\:
    \begin{equation}
      0 < \chi _k \leq < 1
    \end{equation}
    当$\chi _{uk}$接近于零时,输入完全饱和,$\chi _{uk}$指的是完全不饱和。第K辆车的动态方程可以进一步写为\:
    \begin{eqnarray}
      &\dot{x}_k(t) =& v_i(t) \nonumber \\
      &\dot{v}_k(t) =& a_i(t) \nonumber \\
      &\dot{a}_k(t) =& f_k(v_k,a_k) \nonumber + \frac{\chi _{uk}\omega (t) u_k(t) }{\tau _k} + \epsilon _k(t) + \frac{r_k(t)}{\tau _k} \nonumber \\
    \end{eqnarray}
    令干扰$d_k(t) = \epsilon (t) + \frac{r_K(t)}{\tau _K}$。
    \paragraph{假设1}制动器的功效$\omega (t)$,制动器的偏差$\r_k(t)$和由阵风或者路面情况而造成的额外干扰$\epsilon _k$
    都是未知函数,并且存在正数$\omega_{k0}$和$L_0$满足$<\omega _{k0} \ge\omega _K(t) \ge 1$,$d_K(t) < L_0 <\infty$。
    \paragraph{假设2}干扰的一阶导数,二阶导数存在,并且$|\dot{d}|<L_1,|\ddot{d}|<L_2$,其中$\dot{d}$的上确界$L_1$被假设
    未知,但是$\ddot{d}$的上确界是已知的,并且$L_1,L_2$是有限的。
    \paragraph{注}制动器的故障和饱和特性没有被解决,下一部分将会设计故障允许控制器完成对它们的控制。不过由于控制器有较长的训练时间,
    所以这个方法不能用于处理密集的交通情况下。

    



    \subsection{数学知识}

    
    \paragraph{引理1}
    由于RBFNNs本身的逼近能力,用它来估计未知的非线性函数。任何的连续函数$f(Z)$可以被替代为\:
    \begin{equation}
      f(Z) = W^{*T} \pi(Z) + \zeta 
    \end{equation}
    其中Z是输入向量,$W^* = {[\omega_1, \omega_2, ... ,\omega_n ]}^T$是n个神经元理想权重,\zeta 为逼近误差,
    $\pi(Z) = [\pi _1(Z),\pi_2 (Z),...,\pi(Z)]^T$是高斯基向量函数。可以表示为\:
    \begin{equation}
      \pi _K(Z) = \mathcal{exp}[-\frac{{(Z-C_k)}^T(Z-C_k)}{2{b_k}^2}], \quad k = 1,2 ,..., n
    \end{equation}
    其中$b_k$是高斯函数的宽度,$C_k$是中心向量。

    \paragraph{引理2}
    下面是一个一阶滑膜控制器的原型,对于这个系统存在以下的动态方程:  
    \begin{equation}
      \dot{s}_a(t) = a(t) + u_a(t)
    \end{equation}  
    其中$a(t)$为不确定度(干扰),并且满足$|a|<a_0,|\dot{a}|<a_1$,其中$a_0$是未知的,而$a_1$
    是已知的。控制输入的表达式为:
    \begin{equation}
      u_a(t) = -(h_a(t) +\delta _a)sgn(s_a)
    \end{equation}
    其中$\delta _a> 0 $并且$h_a(t)$满足以下{{red}两层适应率}。
    \begin{eqnarray}
      &\dot{h}_a(t) &= - \varphi_a (t) sgn(\eta  _a(t))  \\
      &\dot{r}_a(t) &= \lambda  |\eta_a (t)| + r_0 \sqrt{\lambda  }sgn(e_a(t))
    \end{eqnarray}
    其中\:
    \begin{eqnarray}  
      &\varphi_a (t) &= r_{0a} + r_a(t) \\
      &e_a(t)& = q a_1/\alpha  - r_a(t) \\
      %&\dot{r}_a(t) &= \lambda |\eta (t)| + r_{0a}\sqrt{\lambda }sgn(e_a(t)) \\
      &\eta  (t) &= h_a(t) - \frac{1}{\alpha_a}|\bar{u}_{eq}| - \beta \\
      &\dot{\bar{u}}_{eq-a} &= \frac{1}{\tau _a}(-(\delta_a + h_a )sgn(s_a)(t)-\bar{u}_{eq-a})
    \end{eqnarray}
    如果$0<\alpha <1,q>1,\lambda ,\beta >0$,那么$h_a(t)>|a(t)|$可以在有限的时间内成立。
    进而$\dot{s}_a(t)<0$,可推出状态变量可以进入滑动状态,最后{{red}$\varphi_a (t),\dot{h}_a(t)$有界}。
    
    \paragraph{注释}
    在滑模面$s_a$上,等效控制输入$u_{eq-a}$为$a(t)$。通过(6)-(10),$h_a$一直增加直到,直到运动
    到滑膜上。然后$h_a$开始递减收敛到{{red}安全的边界}。$u_{eq-d}$的边界满足以下方程\:
    \begin{equation}
      h_a(t) \cong \frac{1}{\alpha _a}\bar{u}_{eq}(t)| -\beta 
      =\frac{1}{\alpha _a}\bar{u}_{eq}(t)| -\beta
    \end{equation}
    
    
    
\section{控制器的设计}

\subsection{自适应扰动观测器设计(ASMDO)}
  \paragraph{}为了对系统去设计一个误差观测器,引入辅助变量如下:
  \begin{equation}
    e = z - x_n  
  \end{equation}
  其中$\dot{z}$满足下面等式:
  \begin{eqnarray}
    \dot{z} = &f(x) + g(x)U(t) - n_{1d} sgn(e) \left|e \right|^{\varTheta_{1n}} \nonumber \\
    & -n_{2d} sgn(e)\left|e\right|^{\varTheta_{2n}} +\upsilon_z 
  \end{eqnarray}
  系数$n_{1d}$,$n_{2d}$,$\theta _{1d}$和$\theta _{1n}$是正常数并且$\theta _{1n}$<1以及$\theta _{2n}$>1.参考下面的引理,
  设计$\upsilon_z$使我们设计的辅助变量$e$和干扰$d$的估计值$\hat{d}$在有限的时间内收敛到零。

\subsubsection{定理1}
  引入如下滑模面$s_d$:
  \begin{equation}
    s_d = \dot{e} + n_{1d} sgn(e) \left| e \right|^{\varTheta_{1n}} + n_{2d} sgn(e)\left|e\right|^{\varTheta_{2n}} 
  \end{equation}
  如果$\upsilon_z$满足下面动态方程:
  \begin{equation}
    \dot{\upsilon}_z = -(\delta_d + h_d) sgn(s_d),\qquad \upsilon_z(0) = 0 
  \end{equation}
  其中如果$\delta_d$是一个正数并且$h_d$满足$h_d>L_1$,那么e在有限的时间内收敛到0并且对误差估计值$\bar{d}$将满足\:
  \begin{equation}
    \hat{d} = \upsilon_z
  \end{equation} 
  
  \paragraph*{证明}
    首先对$e$求导:
    \begin{equation}
      \dot{e} = \dot{z}-\dot{x}_n =  n_{1d} sgn(e) \left| e \right|^{\varTheta_{1n}} + n_{2d} sgn(e)\left|e\right|^{\varTheta_{2n}} 
    \end{equation}
    将(9)带入(6)可\:
    \begin{equation}
      s_d = \upsilon _z -d
    \end{equation}
    将(7)中$\dot{\upsilon }_z$带入计算可达到$\dot{s}_d$\:
    \begin{equation}
      \dot{s}_d = \dot{\upsilon }_z - \dot{d} = -(\delta + h_d)sgn(s_d) - \dot{d}
    \end{equation}
    引入Lyapunov方程为$V_d = s_d ^2 //2$,结合(11)可得\:
    \begin{eqnarray}
      &\dot{V}_d &= s_d \dot{s}_d = -(\delta + h_d)\left| s_d \right| - \dot{d}s_d \nonumber\\
      & &  \leq (\left| {\dot{d}} \right| -h_d)  \left| s_d \right| - \delta \left| s_d \right|
    \end{eqnarray}
    由于$\delta>$ 和$h_d>L_1$,所以可得\:
    \begin{equation*}
      s_d \dot{s}_d \leq \delta \left| s_d \right| \\    
    \end{equation*}
    通过Lyapunov方程我们可以推出在有限的时间下滑模面$s_d$ 收敛于零,并且我们可以得到\:
    \begin{equation}
      \dot{e} = - n_{1d}sgn(e)\left| e\right|^{\theta_{1d}} - n_{2d}sgn(e)  \left| e\right|
    \end{equation}
    通过引理(),(13)式中的$\dot{e}$在有限的时间内,趋向于零。

    假设变量为$e_d$是误差与误差估计值的差值:
    \begin{equation}
      e_d = d - \hat{d}
    \end{equation}
    联立(8),(10),可得$e_d = 0$,最终上述的观测器在有限的时间内可以跟踪上干扰$d$。 \\
    在上述引理中,如果要使得$\hat{d}$能够估计$d$,需要满足$h_d > L_0$。但是在汽车模型中, $h_d$一般是未知的。如果我们把$h_d$
    的值设计的太大,将会造成滑膜面的抖动,控制器效果变差。在下面一个定理中,我们将对$h_d$进行设计。
    
  \subsubsection{定理2}
    \paragraph{}
    采用两层适应率增益$h_d$可为:
    \begin{eqnarray}
      &\dot{h}_d(t) &= - \varphi_d (t) sgn(\eta  _d(t))  \\
      &\dot{r}_d(t) &= \kappa |\eta_d (t)| + r_0 \sqrt{\kappa }sgn(e(t)_d)
    \end{eqnarray}
    其中:
    \begin{eqnarray}
      &\varphi_d (t) &= r_{0d} + r_d(t) \\
      &e_d(t)& = q d_1/\alpha  - r_d(t) \\
      %&\dot{r}_d(t) &  = \lambda |\eta (t)| + r_{0d}\sqrt{\lambda }sgn(e_d(t)) \\
      &\eta_d  (t) &= h_d(t) - \frac{1}{\alpha_d}|\bar{u}_{eq-d}| - \beta \\
      &\dot{\bar{u}}_{eq-d} &= \frac{1}{\tau _d}(-(\delta_d + h_d )sgn(s_d)(t)-\bar{u}_{eq-d})
    \end{eqnarray}
    如果$0 <\alpha <1$,\quad $\beta >0$,\quad $r_0>0$,\quad$q>1$成立,那么$h_d > |L_1|$可以在有限的时间内满足,
    从而使得系统能运行在滑膜状态,进一步使得$h_d(t)$和$\varphi(t)$是有界的。
    
  \paragraph*{证明}
    根据假设(1),干扰$d(t)$的一阶和二阶导数是有界的。用$-\dot{b}(t),s_d$替换引理(1)中的$a(t),s_a$,
    可以得到通过在式子()(){\color{red}二阶适应率}作用下,可以估计误差$d(t)$。
  \paragraph*{说明}
    由于在智能汽车模型中一般不能得到$x_n$,但是我们根据

  \subsection{滑膜控制器的设计}
    这部分的目标是建立一个鲁棒性良好的滑模面控制方法,误差观测器也被用来去估计误差。\\
    假设需要追踪的轨迹是$y_d(t)$,那么追踪误差可为$e_m = y(t) - y_d(t)$。首先,设计滑膜面为\:
    \begin{equation}
      s_c(t) = \int_{0}^{t} (\gamma_a{e_m}^\eta(t) + \gamma _z e_m(t)) \,dx
      + e_m(t) + \dot{e}_m(t)
    \end{equation}
    其中$\eta > 0 ,\gamma _a > 0,\gamma _z >0 $。$\gamma_a,\gamma _z$是两个自适应参量,其表达式为\:
    \begin{eqnarray*}
      \gamma _a &=& \frac{m_a}{m_a + n_a \mathcal{exp}(-c_a|v_z(t)|)} \nonumber \\
      \gamma _z &=& \frac{1}{1+\mathcal{exp}(-c_z|e_m(t)|)}\nonumber
    \end{eqnarray*}
    其中$m_a > 0,n_a >0 , c_a >0, c_z >0$。
    \paragraph{}引进这两个自适应参数是为了减弱集中干扰并且提升系统的收敛性。当干扰增大时,干扰的估计值$v_z(t)$也增大,从而可以推出
    $\gamma _a$占比上升,补偿干扰的影响。同理,当跟踪误差变大时,$\gamma _z $增加。因此系统的收敛性和鲁棒性都增加。
    \paragraph{}联立()()可得\:
    \begin{equation}
      s_c(t) = \int_{0}^{t} (\gamma_a{e_m}^\eta(t) + \gamma _z e_m(t))  \,dx + e_m(t) 
      + x_2(t) - \dot{y}(t)
    \end{equation}
    对上式求导可得\:
    \begin{equation}
      \dot{s}_c(t) = \gamma_a{e_m}^\eta(t) + \gamma _z e_m(t) + \dot{e}_m(t) + \dot{x}_2(t) - \ddot{y}(t)
    \end{equation}
    
  % \\
  % 推导过程:\\
  % 1.假设\\
  % 2.数学推导\\
  % 3.证明\\

 




  %数学模型->控制器设计->稳定性分析->实验器材介绍;

\section{结果与讨论}
描述结果\:\\
视觉+语言:定性与定量\\
分析结构:论证+对比\\
讨论结果:意义/展望/猜想\(详略得当\)\\


\begin{theorem}
  定理内容。
  “定义”、“假设”、“公理”、“引理”等的排版格式与此相同,详细定理证明、公式可放在附录中。
\end{theorem}


\begin{proof}
  证明过程.
\end{proof}

\begin{figure}[htb]
  \centering
  \includegraphics[width=\linewidth]{example-fig.pdf}
  \caption{图片说明 *字体为小 5 号,图片应为黑白图,图中的子图要有子图说明*}
\end{figure}

\begin{table}[htb]
  \centering
  \caption{表说明}
  \small
  \begin{tabular}{cc}
    \toprule
    示例表格 & 第一行为表头,表头要有内容 \\
    \midrule
    & \\
    \midrule
    & \\
    \bottomrule
  \end{tabular}
\end{table}

% \begin{procedure}
%   \caption{过程名称}
%   \small
%   \begin{algorithmic}
%     \REQUIRE
%     \ENSURE
%     \STATE \COMMENT{《计算机学报》的方法过程描述字体为小5号宋体,IF 、THEN等伪代码关键词全部用大写字母,变量和函数名称用斜体}
%   \end{algorithmic}
% \end{procedure}

% \begin{algorithm}
%   \caption{算法名称}
%   \small
%   \begin{algorithmic}
%     \REQUIRE $n \geq 0 \vee x \neq 0$
%     \ENSURE $y = x^n$
%     \STATE $y \leftarrow 1$
%     \IF{$n < 0$}
%       \STATE $X \leftarrow 1 / x$
%       \STATE $N \leftarrow -n$
%     \ELSE
%       \STATE $X \leftarrow x$
%       \STATE $N \leftarrow n$
%     \ENDIF
%     \WHILE{$N \neq 0$}
%       \IF{$N$ is even}
%         \STATE $X \leftarrow X \times X$
%         \STATE $N \leftarrow N / 2$
%       \ELSE[$N$ is odd]
%         \STATE $y \leftarrow y \times X$
%         \STATE $N \leftarrow N - 1$
%       \ENDIF
%     \ENDWHILE
%   \end{algorithmic}
% \end{algorithm}



% \begin{acknowledgments}
%   致谢内容。
% \end{acknowledgments}


% \nocite{*}

% \bibliographystyle{cjc}
% \bibliography{example}


% \newpage

% \appendix
\section{结论}


% 附录内容置于此处,字体为小5号宋体。附录内容包括:详细的定理证明、公式推导、原始数据等


% \makebiographies


% \begin{background}
% *论文背景介绍为英文,字体为小5号Times New Roman体*

% 论文后面为400单词左右的英文背景介绍。介绍的内容包括:

% 本文研究的问题属于哪一个领域的什么问题。该类问题目前国际上解决到什么程度。

% 本文将问题解决到什么程度。

% 课题所属的项目。

% 项目的意义。

% 本研究群体以往在这个方向上的研究成果。

% 本文的成果是解决大课题中的哪一部分,如果涉及863/973以及其项目、基金、研究计划,注意这些项目的英文名称应书写正确。
% \end{background}

\begin{thebibliography}{1}
  \bibitem{liu} 刘海洋. \LaTeX 入门 [M]. 北京: 电子工业出版社, 2013.
  \bibitem{hu}  胡伟. \LaTeX 2e完全学习手册(第二版). 北京: 清华大学出版社, 2013.
  \end{thebibliography}


\end{document}
